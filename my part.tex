\documentclass[12pt,a4paper]{article}
\begin{document}
\begin{titlepage}
    \begin{center}
\line(1,0){500}\\
\huge{\bfseries A CONCEPT PAPER ABOUT BILATERAL TRADE IN UGANDA\\}
\line(1,0){500}\\
[0.25in]
\huge{\bfseries A CONCEPT PAPER SUBMITTED TO THE SCHOOL OF COMPUTING AND INFORMATICS TECHNOLOGY}\\
[1.5in]
\line(1,0){300}\\
\huge{\bfseries Group Members}
\line(1,0){300}\\
TAGUYISIZA YAHAYA|14/U/15104/PS\\
KATAMBA SAMUEL|14/U/23285/PS\\
COLLIN NUWA KAJUBI |14/U/314\\
SAUL MAWANDA |14/U/22411/PS\\
\end{center}
\end{titlepage}
\author{GROUP 122}
\section{Background}
According to Elias Koutsoupias, a supervisor in the Department of Computer at the University of Oxford, one of the fundamental problems at the interface of algorithms and economics is the problem of designing simple algorithms for simple trade problems that are incentive compatible[3]. Incentive compatibility, state in game theory and economics that occurs when the incentives that motivate the actions of individual participants are consistent with following rule established by the group[4]. In this case the simple trade problem to tackle is to investigate implementing of optimal algorithms for bilateral trade. Bilateral trade is the exchange of goods between two countries that facilitates trade and investment by reducing or eliminating tariffs, import quotas, export restraints and other barriers. [1] In this paper we look for and highlight algorithms that can enable both countries taking part in bilateral trade to gain positively from the trade.
\section{Problem statement}
Some of the algorithms that available are not incentive compatible or sufficient as they do not encourage countries to take part in bilateral trade. That is one country always seems to gain more from bilateral trade than the other. Thus the that seems to be country losing more tends not to fully take part in the bilateral agreement and at times opts to terminate the deal or agreement.
\section{Aim and Objectives}
\subsection{General Objectives}
The investigation is intended to find and design optimal algorithms bilateral trade problems in Uganda
\subsection{Specific Objectives}
\begin{itemize}
\item To find out buyer and seller activities involved in bilateral trade in Uganda.
\item To collect and gather data and information on how bilateral trade is done in Uganda.
\item To design optimal algorithms that will implemented in solving the problems facing bilateral trade in Uganda.
\end{itemize}

\section{Scope of Study}
\subsection{Geographical Scope}
The study is to be conducted along Uganda borders with other Countries that border it like Kenya, Tanzania, Democratic Republic of Congo, Rwanda and South Sudan.

\subsection{Theoretical Scope}
The study mainly about gathering information on the main performance factors of bilateral trade and carrying out analysis on information gathered so as to come up with optimal algorithms.

\section{Significance of Study}
\begin{itemize}
\item The study is important in the way that it will help different authorities in implementing viable solutions that will help enhance bilateral trade in Uganda more effectively.
\item The study’s importance is to find out the key performance indicators of bilateral trade in Uganda and using those indicators to find, and design optimal algorithms that will solve problems facing bilateral trade.
\end{itemize}
\section{References}
\begin{bibliography}
[[1] 	"Bilateral Trade," Investopedia, [Online]. Available: www.investopedia.com/terms/b/bilateral-trade.asp. [Accessed 12 April 2017].
	
[2] 	"Algorithm," TechTerms, [Online]. Available: www.techterms.com/definition/algorithm. [Accessed 12 April 2017].
	
[3] 	E. Koutsoupias, "Student Projects," University of Oxford, [Online]. Available: www.cs.ox.ac.uk/teaching/studentprojects/563.html. [Accessed 18 April 2017].
	
[4] 	J. Harvey S. James, "Incentive compatibility," Encyclopedia Britannica, Inc, [Online]. Available: https://global.britannica.com/topic/incentive-compatibility. [Accessed 18 April 2017].	
\end{bibliography}

\end{document}  
